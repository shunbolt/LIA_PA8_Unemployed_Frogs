%% Modèle / template pour début de document via Texmaker / Latex %% 

% By Raphaël Tran %

% Type de document% 
\documentclass{article}

%% Package algorithmique (traduit en français) %% 

\usepackage{algorithmic}
\usepackage{algorithm}
\renewcommand{\algorithmicrequire}{\textbf{Donnee}}
\renewcommand{\algorithmicensure}{\textbf{Variables locales}}
\renewcommand{\algorithmicend}{\textbf{fin}}
\renewcommand{\algorithmicif}{\textbf{si}}
\renewcommand{\algorithmicthen}{\textbf{alors}}
\renewcommand{\algorithmicelse}{\textbf{sinon}}
\renewcommand{\algorithmicelsif}{\algorithmicelse\ \algorithmicif}
\renewcommand{\algorithmicendif}{\algorithmicend\ \algorithmicif}
\renewcommand{\algorithmicfor}{\textbf{pour}}
\renewcommand{\algorithmicdo}{\textbf{faire}}
\renewcommand{\algorithmicendfor}{\algorithmicend\ \algorithmicfor}
\renewcommand{\algorithmicwhile}{\textbf{tant que}}
\renewcommand{\algorithmicendwhile}{\algorithmicend\ \algorithmicwhile}
\renewcommand{\algorithmicprint}{\textbf{afficher}}
\renewcommand{\algorithmicreturn}{\textbf{retourner}}
\renewcommand{\algorithmictrue}{\textbf{vrai}}
\renewcommand{\algorithmicfalse}{\textbf{faux}}
\renewcommand{\listalgorithmname}{Liste des \ALG@name s}
\newcommand{\LINEFOR}[2]{%
    \STATE\algorithmicfor\ {#1}\ \algorithmicdo\ {#2} %
}
\newcommand{\LINEIF}[2]{%
	\STATE\algorithmicif\  {#1}\ \algorithmicthen\  {#2} %
}
\algsetup{indent=1cm}



% Package divers % 
\usepackage[english]{babel} % Langage francais
\usepackage{graphicx} % Images
\usepackage[utf8]{inputenc} % Police
\usepackage{amsmath} % Langage mathématique
\usepackage[top=3cm, bottom=2cm, left=4cm, right=4cm]{geometry} % Marges
\usepackage{amssymb} 
\usepackage{amsthm}
\usepackage[x11names]{xcolor}
\usepackage{variations}
\usepackage{fancybox}
\usepackage{comment} % multiline comment
\usepackage{wrapfig} % Text next to images
\usepackage{lipsum} % fill text
\usepackage{hyperref} % hyperlink 
\usepackage{eurosym} % euro 
\usepackage{varwidth} % blocs de texte
\usepackage{kbordermatrix} % labeled matrix
\usepackage{epigraph} % epigraph et citations
\usepackage{titlesec}
\usepackage{emptypage} 
%\usepackage[backend = biber]{biblatex} % bibliography
%\addbibresource{eton.bib}
\usepackage{csquotes}
\usepackage{CJKutf8}


% Package de dessin tikz aka best package ever%

\usepackage{tikz} 
\usetikzlibrary{arrows,3d}


%Package de tête et pied de page \fancyhdr 
\usepackage{fancyhdr}
\pagestyle{fancy}
\fancyhf{}
\rhead{}
\lhead{\rightmark}
\cfoot{\thepage}
\lfoot{Tran Raphaël}
\rfoot{M1-BD \it{2019-2020}}

% Titre de première page %
\newcommand{\HRule}{\rule{\linewidth}{0.5mm}}


\makeindex
\begin{document}
\begin{titlepage}
\center

 \LARGE \textsc{Efrei Paris}
\\[1.5cm]
\Large \textsc{Andréas TOPP}
\\[0.5cm]
\large \textsc{Cours}
\\[0.5cm]
\HRule \\[0.4cm]
 {\Huge \bfseries Gestion de projet}\\[0.4cm] % Title of your document
\HRule \\[1.4cm]

\includegraphics[scale=1]{../../../Logo_Efrei.jpg}  \\[1cm]

{\Large Raphaël \textsc{Tran}}  \\[0.5cm]
{\large \today } \\[2pt]

\end{titlepage}

\tableofcontents
\newpage

\section{Méthodologies et concept}

\subsection{Synthèse écrite d'idées innovantes}

Pour (cible) qui souhaitent (besoin) notre produit est (nature prod/ service) qui (bénéfice/utilité) à la différence de (solution actuelle, concurrence)  permet de (éléments différentiateurs). 

\medbreak

Pour les \textbf{adolescents (12-18 ans) victimes de harcèlements et violences à l'école ou au domicile} qui souhaitent \textbf{signaler / en parler}, notre produit est \textbf{un chatbot} qui \textbf{permet de dialoguer avec les enfants victimes de violences physiques ou verbales} à la différence des {\bf  ligne téléphoniques anti-harcèlement} permet d' \textbf{avoir un moyen de communication plus accessible et plus agréable pour l'enfant.} 

\textit{Bien. A condenser un peu pour mieux vendre le projet et faire passer le message. Définir le chatbot auprès des vendeurs}

\subsection{Parties prenantes du projet}

Lister les différentes parties prenantes du projet. 

\begin{itemize}
\item \textbf{Enfants victimes :} Personnes physiques. Principaux utilisateurs front du bot. 
\item \textbf{Écoles :} Personnes morales. Clients et utilisateurs back du programme. Destinataire des alertes et des fiches.
\item \textbf{Psychologues :} Personnes physiques. Conseille et adapte l'interaction entre la victime et le bot.
\item \textbf{Serveur :} Personne morale. Permet le stockage des données et éventuellement le test de l'algorithme.   
\end{itemize}



\end{document}
